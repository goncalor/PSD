\begin{table}[h]
\centering
\begin{tabular}{|c||c|c|c|c|c|}
\hline
 & d & e & f & g & h  \\
\hline
\hline
e & $-$ & $\times$ & $\times$ & $\times$ & $\times$  \\
\hline
f & $-$ & $-$ & $\times$ & $\times$ & $\times$ \\
\hline
g & $-$ & \checkmark & \checkmark & $\times$ & $\times$  \\
\hline
h & \checkmark & $-$ & \checkmark & $-$ & $\times$ \\
\hline
i & \checkmark & \checkmark & $-$ & $-$ & $-$ \\
\hline
\end{tabular}
\caption{Produtos entre constantes a ler da memória}
\label{tab:produtos}
\end{table}

De maneira a reduzir ao máximo o número de ciclos que o algoritmo leva, é adequado agrupar os pares de leitura da memória de maneira a que o produto seja feito e armazenado ao mesmo tempo da leitura.

Uma operação é feita assim que os seus precedentes ficam disponíveis, quer via registos, quer via leitura directa da memória.

Uma vez que o número de operandos a ler é ímpar ($9$), e como o objectivo é maximizar o \emph{throughput}, haverá dois tipos distintos de execução.

\subsubsection{Execuções ímpares (1ª, 3ª, ...)}

\begin{enumerate}
\item %1

\begin{table}[H]
\centering
\begin{tabular}{l|c|c}
Operação & \multicolumn{2}{c}{Operandos} \\
\hline
\texttt{READ} & $d$ & $h$ \\
\hline
\texttt{MUL} & $d$ & $h$ \\
\hline
\texttt{STORE} & \multicolumn{2}{c}{$R_1\leftarrow d$} \\
\texttt{STORE} & \multicolumn{2}{c}{$R_2\leftarrow h$} \\
\texttt{STORE} & \multicolumn{2}{c}{$R_3\leftarrow dh$} \\
\end{tabular}
\caption{Operações do 1º ciclo de execução ímpar}
\label{tab:odd_1}
\end{table}

\item %2

\begin{table}[H]
\centering
\begin{tabular}{l|c|c}
Operação & \multicolumn{2}{c}{Operandos} \\
\hline
\texttt{READ} & $i$ & $e$ \\
\hline
\texttt{MUL} & $i$ & $e$ \\
\texttt{MUL} & $i$ & $R_1$($d$) \\
\hline
\texttt{STORE} & \multicolumn{2}{c}{$R_1\leftarrow di$} \\
\texttt{STORE} & \multicolumn{2}{c}{$R_4\leftarrow e$} \\
\texttt{STORE} & \multicolumn{2}{c}{$R_5\leftarrow ei$} \\
\end{tabular}
\caption{Operações do 2º ciclo de execução ímpar}
\label{tab:odd_2}
\end{table}

\item %3

\begin{table}[H]
\centering
\begin{tabular}{l|c|c}
Operação & \multicolumn{2}{c}{Operandos} \\
\hline
\texttt{READ} & $f$ & $g$ \\
\hline
\texttt{MUL} & $f$ & $g$ \\
\texttt{MUL} & $f$ & $R_2$($h$) \\
\texttt{MUL} & $g$ & $R_4$($e$) \\
\hline
\texttt{STORE} & \multicolumn{2}{c}{$R_6\leftarrow fg$} \\
\texttt{STORE} & \multicolumn{2}{c}{$R_2\leftarrow fh$} \\
\texttt{STORE} & \multicolumn{2}{c}{$R_4\leftarrow eg$} \\
\end{tabular}
\caption{Operações do 3º ciclo de execução ímpar}
\label{tab:odd_3}
\end{table}

\item %4

\begin{table}[H]
\centering
\begin{tabular}{l|c|c}
Operação & \multicolumn{2}{c}{Operandos} \\
\hline
\texttt{READ} & $a$ & $b$ \\
\hline
\texttt{SUB} & $R_5$($ei$) & $R_2$($fh$) \\
\texttt{SUB} & $R_6$($fg$) & $R_1$($di$) \\
\texttt{SUB} & $R_3$($dh$) & $R_4$($eg$) \\
\hline
\texttt{STORE} & \multicolumn{2}{c}{$R_1\leftarrow a$} \\
\texttt{STORE} & \multicolumn{2}{c}{$R_2\leftarrow b$} \\
\texttt{STORE} & \multicolumn{2}{c}{$R_4\leftarrow ei - fh$} \\
\texttt{STORE} & \multicolumn{2}{c}{$R_5\leftarrow fg - di$} \\
\texttt{STORE} & \multicolumn{2}{c}{$R_6\leftarrow dh - eg$} \\
\end{tabular}
\caption{Operações do 4º ciclo de execução ímpar}
\label{tab:odd_4}
\end{table}

\item %5

\begin{table}[H]
\centering
\begin{tabular}{l|c|c}
Operação & \multicolumn{2}{c}{Operandos} \\
\hline
\texttt{READ} & $c$ & $c_{n+1}$ \\
\hline
\texttt{MUL} & $R_4$($ei - fh$) & $R_1$($a$) \\
\texttt{MUL} & $R_5$($fg - di$) & $R_2$($b$) \\
\texttt{MUL} & $R_6$($dh - eg$) & $c$ \\
\hline
\texttt{STORE} & \multicolumn{2}{c}{$R_4\leftarrow a(ei - fh)$} \\
\texttt{STORE} & \multicolumn{2}{c}{$R_5\leftarrow b(fg - di)$} \\
\texttt{STORE} & \multicolumn{2}{c}{$R_6\leftarrow c(dh - eg)$} \\
\texttt{STORE} & \multicolumn{2}{c}{$R_8\leftarrow c_{n+1}$} \\
\end{tabular}
\caption{Operações do 5º ciclo de execução ímpar}
\label{tab:odd_5}
\end{table}

\item %6

\begin{table}[H]
\centering
\begin{tabular}{l|c|c}
Operação & \multicolumn{2}{c}{Operandos} \\
\hline
\texttt{ADD} & $R_4$($a(ei - fh)$) & $R_5$($b(fg - di)$) \\
\hline
\texttt{STORE} &\multicolumn{2}{c}{$R_7\leftarrow a(ei - fh) + b(fg - di)$} \\
\end{tabular}
\caption{Operações do 6º ciclo de execução ímpar}
\label{tab:odd_6}
\end{table}

\item %7

\begin{table}[H]
\centering
\begin{tabular}{l|c|c}
Operação & \multicolumn{2}{c}{Operandos} \\
\hline
\texttt{ADD} & $R_7$ & $R_6$($c(dh - eg)$) \\
\hline
\texttt{STORE} &\multicolumn{2}{c}{$R_7\leftarrow \left|\begin{matrix}a&b&c\\d&e&f\\g&h&i\end{matrix}\right|$} \\
\end{tabular}
\caption{Operações do 6º ciclo de execução ímpar}
\label{tab:odd_6}
\end{table}

\end{enumerate}

\subsubsection{Execuções pares (2ª, 4ª, ...)}
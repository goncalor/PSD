\begin{table}[h]
\centering
\begin{tabular}{|c||c|c|c|c|c|}
\hline
 & d & e & f & g & h  \\
\hline
\hline
e & $-$ & $\times$ & $\times$ & $\times$ & $\times$  \\
\hline
f & $-$ & $-$ & $\times$ & $\times$ & $\times$ \\
\hline
g & $-$ & \checkmark & \checkmark & $\times$ & $\times$  \\
\hline
h & \checkmark & $-$ & \checkmark & $-$ & $\times$ \\
\hline
i & \checkmark & \checkmark & $-$ & $-$ & $-$ \\
\hline
\end{tabular}
\caption{Produtos entre constantes a ler da memória}
\label{tab:produtos}
\end{table}

De maneira a reduzir ao máximo o número de ciclos que o algoritmo leva, é adequado agrupar os pares de leitura da memória de maneira a que o produto seja feito e armazenado ao mesmo tempo da leitura.

Uma operação é feita assim que os seus precedentes ficam disponíveis, quer via registos, quer via leitura directa da memória.

Uma vez que o número de operandos a ler é ímpar ($9$), e como o objectivo é maximizar o \emph{throughput}, haverá dois tipos distintos de execução.

\subsubsection{Execuções ímpares (1ª, 3ª, ...)}

\begin{enumerate}
\item %1
\begin{table}[H]
\centering
\begin{tabular}{l|c|c}
Operação & \multicolumn{2}{|c}{Operandos} \\
\hline
\texttt{READ} & $d$ & $h$ \\
\hline
\texttt{MUL} & $d$ & $h$ \\
\hline
\texttt{STORE} & $R_1$ & $d$ \\
\texttt{STORE} & $R_2$ & $h$ \\
\texttt{STORE} & $R_3$ & $dh$ \\
\end{tabular}
\caption{Operações do 1º ciclo de execução ímpar}
\label{tab:odd_1}
\end{table}

\item %2
\begin{table}[H]
\centering
\begin{tabular}{l|c|c}
Operação & \multicolumn{2}{|c}{Operandos} \\
\hline
\texttt{READ} & $i$ & $e$ \\
\hline
\texttt{MUL} & $i$ & $e$ \\
\texttt{MUL} & $i$ & $R_1$($d$) \\
\hline
\texttt{STORE} & $R_1$ & $di$ \\
\texttt{STORE} & $R_4$ & $e$ \\
\texttt{STORE} & $R_5$ & $ie$ \\
\end{tabular}
\caption{Operações do 2º ciclo de execução ímpar}
\label{tab:odd_2}
\end{table}

\item %3
\begin{table}[H]
\centering
\begin{tabular}{l|c|c}
Operação & \multicolumn{2}{|c}{Operandos} \\
\hline
\texttt{READ} & $f$ & $g$ \\
\hline
\texttt{MUL} & $f$ & $g$ \\
\texttt{MUL} & $f$ & $R_2$($h$) \\
\texttt{MUL} & $g$ & $R_4$($e$) \\
\hline
\texttt{STORE} & $R_6$ & $fg$ \\
\texttt{STORE} & $R_2$ & $fh$ \\
\texttt{STORE} & $R_4$ & $ge$ \\
\end{tabular}
\caption{Operações do 3º ciclo de execução ímpar}
\label{tab:odd_3}
\end{table}


\end{enumerate}

\subsubsection{Execuções pares (2ª, 4ª, ...)}
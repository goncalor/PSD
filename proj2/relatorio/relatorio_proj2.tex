\documentclass[a4paper]{article}

\usepackage[portuguese]{babel}
\usepackage{comment}
\usepackage[T1]{fontenc}
\usepackage[utf8]{inputenc}
\usepackage{hyperref}
\usepackage{graphicx}
\usepackage{float}
\usepackage{multirow}
\usepackage[hypcap]{caption} % makes \ref point to top of figures and tables
\usepackage{amsmath}
\usepackage[usenames,dvipsnames,svgnames,table]{xcolor}

\begin{document}

\begin{titlepage}

	\begin{center}

		\includegraphics[width=6cm]{./title}\\[3cm]

		\textsc{\LARGE Projecto de Sistemas Digitais}\\[1.5cm]

		\textsc{\Large Laboratório 3}\\[1.5cm]


		{ \huge \bfseries Processador Dedicado para \\[3mm] Processamento de Imagens Binárias \\[3cm] }


		\noindent
		\begin{minipage}{0.4\textwidth}
			\begin{flushleft} \large
				Rafael Gonçalves, 73786
			\end{flushleft}
		\end{minipage}
		\begin{minipage}{0.4\textwidth}
			\begin{flushright} \large
				Gonçalo Ribeiro, 73294
			\end{flushright}
		\end{minipage}\\[2.0cm]

		\large \textit{Docente: Prof. Horácio Neto}

		\vfill

		{\large \today}


	\end{center}

\end{titlepage}

\tableofcontents
\pagebreak

\section{Optimização do Algoritmo}
De maneira a minimizar o número de multiplicações utilizado no algoritmo, reorganizámos a expressão aritmeticamente.

\[
\left| A \right| = \left|
\begin{matrix}
a & b & c\\
d & e & f\\
g & h & i
\end{matrix}
\right| \Leftrightarrow\]

\[
\Leftrightarrow \left| A \right| = a e i - a f h - b d i + c d h + b f g - c e g \Leftrightarrow
\]

\[
\Leftrightarrow \left| A \right| = a \cdot ( e i - f h ) + b \cdot ( f g - d i ) + c \cdot ( d h - e g )
\]

Isto é, passámos de um algoritmo com 12 produtos para um algoritmo de 9 produtos.

\section{Fluxo de Dados}
\setlength{\unitlength}{0.75mm}
\begin{figure}[H]
\centering
\begin{picture}(140,140)
% Operandos iniciais:
\put(15,130){\makebox(0,0){$e$}}
\put(25,130){\makebox(0,0){$i$}}
\put(35,130){\makebox(0,0){$f$}}
\put(45,130){\makebox(0,0){$h$}}
\put(55,130){\makebox(0,0){$f$}}
\put(65,130){\makebox(0,0){$g$}}
\put(75,130){\makebox(0,0){$d$}}
\put(85,130){\makebox(0,0){$i$}}
\put(95,130){\makebox(0,0){$d$}}
\put(105,130){\makebox(0,0){$h$}}
\put(115,130){\makebox(0,0){$e$}}
\put(125,130){\makebox(0,0){$g$}}
% Setas dos operandos para as operações:
\multiput(15,128)(20,0){6}{\vector(1,-2){2.4}}
\multiput(25,128)(20,0){6}{\vector(-1,-2){2.4}}
% Operações do primeiro nível:
\multiput(20,120)(20,0){6}{\circle{10}}
\multiput(20,120)(20,0){6}{\makebox(0,0){\texttt{x}}}
% Numeração das operações do primeiro nível:
\put(15,115){\makebox(0,0){\texttt{1}}}
\put(35,115){\makebox(0,0){\texttt{2}}}
\put(55,115){\makebox(0,0){\texttt{3}}}
\put(75,115){\makebox(0,0){\texttt{4}}}
\put(95,115){\makebox(0,0){\texttt{5}}}
\put(115,115){\makebox(0,0){\texttt{6}}}
% Setas para o segundo nível:
\multiput(20,115)(20,0){6}{\line(0,-1){15}}
\multiput(20,100)(40,0){3}{\vector(1,0){5}}
\multiput(40,100)(40,0){3}{\vector(-1,0){5}}
% Segundo nível de operações:
\multiput(30,100)(40,0){3}{\circle{10}}
\multiput(30,100)(40,0){3}{\makebox(0,0){\texttt{-}}}
% Numeração das operações do segundo nível:
\put(25,95){\makebox(0,0){\texttt{7}}}
\put(65,95){\makebox(0,0){\texttt{8}}}
\put(105,95){\makebox(0,0){\texttt{9}}}
% Setas para o terceiro nível:
\multiput(30,95)(40,0){3}{\vector(0,-1){10}}
\multiput(45,80)(40,0){3}{\vector(-1,0){10}}
% Terceiro nível de operações:
\multiput(30,80)(40,0){3}{\circle{10}}
\multiput(30,80)(40,0){3}{\makebox(0,0){\texttt{x}}}
% Operadores Constantes do Terceiro Nível:
\put(47,80){\makebox(0,0){$a$}}
\put(87,80){\makebox(0,0){$b$}}
\put(127,80){\makebox(0,0){$c$}}
% Numeração das operações do 3º nível:
\put(25,75){\makebox(0,0){\texttt{10}}}
\put(65,75){\makebox(0,0){\texttt{11}}}
\put(105,75){\makebox(0,0){\texttt{12}}}
% Setas para o 4º e 5º níveis
\multiput(30,75)(40,0){2}{\line(0,-1){15}}
\put(110,75){\line(0,-1){35}}
\put(30,60){\vector(1,0){15}}
\put(70,60){\vector(-1,0){15}}
\put(110,40){\vector(-1,0){35}}
% Quarto nível de operações
\put(50,60){\circle{10}}
\put(50,60){\makebox(0,0){\texttt{+}}}
% Numeração
\put(45,55){\makebox(0,0){\texttt{13}}}
% Seta para o 5º Nível
\put(50,55){\line(0,-1){15}}
\put(50,40){\vector(1,0){15}}
% Quinto nível:
\put(70,40){\circle{10}}
\put(70,40){\makebox(0,0){\texttt{+}}}
% Numeração
\put(65,35){\makebox(0,0){\texttt{14}}}
% Seta para o NOP:
\put(70,35){\vector(0,-1){10}}
% NOP:
\put(70,20){\circle{10}}
\put(70,20){\makebox(0,0){\texttt{NOP}}}
% Prioridades:
\put(75,45){\makebox(0,0){\textcolor{blue}{$1$}}}
\put(55,65){\makebox(0,0){\textcolor{green}{$2$}}}
\put(115,85){\makebox(0,0){\textcolor{green}{$2$}}}
\multiput(35,85)(40,0){2}{\makebox(0,0){\textcolor{yellow}{$3$}}}
\put(115,105){\makebox(0,0){\textcolor{yellow}{$3$}}}
\multiput(35,105)(40,0){2}{\makebox(0,0){\textcolor{orange}{$4$}}}
\multiput(105,125)(20,0){2}{\makebox(0,0){\textcolor{orange}{$4$}}}
\multiput(25,125)(20,0){4}{\makebox(0,0){\textcolor{red}{$5$}}}
\end{picture}
\caption{grafo de fluxo de dados e caminhos críticos}
\label{fig:fluxodados}
\end{figure}

\begin{table}
\centering
\begin{tabular}{|c|c||c l|}
\hline
Operação & Tipo & Prioridade & \\
\hline
\hline
\texttt{1} & \texttt{x} & \textcolor{red}{$5$} & \\
\hline
\texttt{2} & \texttt{x} & \textcolor{red}{$5$} & \\
\hline
\texttt{3} & \texttt{x} & \textcolor{red}{$5$} & \\
\hline
\texttt{4} & \texttt{x} & \textcolor{red}{$5$} & \\
\hline
\hline
\texttt{7} & \texttt{-} & \textcolor{orange}{$4$} & $ \leftarrow $ \texttt{1 , 2} \\
\hline
\texttt{8} & \texttt{-} & \textcolor{orange}{$4$} & $ \leftarrow $ \texttt{3 , 4} \\
\hline
\texttt{5} & \texttt{x} & \textcolor{orange}{$4$} & \\
\hline
\texttt{6} & \texttt{x} & \textcolor{orange}{$4$} & \\
\hline
\hline
\texttt{10} & \texttt{x} & \textcolor{yellow}{$3$} & $ \leftarrow $ \texttt{7} \\
\hline
\texttt{11} & \texttt{x} & \textcolor{yellow}{$3$} & $ \leftarrow $ \texttt{8} \\
\hline
\texttt{9} & \texttt{-} & \textcolor{yellow}{$3$} & $ \leftarrow $ \texttt{5 , 6} \\
\hline
\hline
\texttt{13} & \texttt{+} & \textcolor{green}{$2$} & $ \leftarrow $ \texttt{10 , 11} \\
\hline
\texttt{12} & \texttt{x} & \textcolor{green}{$2$} & $ \leftarrow $ \texttt{9} \\
\hline
\hline
\texttt{14} & \texttt{+} & \textcolor{blue}{$1$} & $ \leftarrow $ \texttt{12 , 13} \\
\hline
\end{tabular}
\caption{lista de prioridades correspondente à Figura~\ref{fig:fluxodados}}
\label{listaprioridades}
\end{table}

\end{document}
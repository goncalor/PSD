\documentclass[a4paper]{article}

\usepackage[portuguese]{babel}
\usepackage{comment}
\usepackage[T1]{fontenc}
\usepackage[utf8]{inputenc}
\usepackage{hyperref}
\usepackage{graphicx}
\usepackage{float}
\usepackage{multirow}
\usepackage[hypcap]{caption} % makes \ref point to top of figures and tables
\usepackage{amsmath}
\usepackage[usenames,dvipsnames,svgnames,table]{xcolor}

\definecolor{yellow}{RGB}{230, 230, 0}
\begin{document}

\begin{titlepage}

	\begin{center}

		\includegraphics[width=6cm]{./title}\\[3cm]

		\textsc{\LARGE Projecto de Sistemas Digitais}\\[1.5cm]

		\textsc{\Large Laboratório 3}\\[1.5cm]


		{ \huge \bfseries Processador Dedicado para \\[3mm] Processamento de Imagens Binárias \\[3cm] }


		\noindent
		\begin{minipage}{0.4\textwidth}
			\begin{flushleft} \large
				Rafael Gonçalves, 73786
			\end{flushleft}
		\end{minipage}
		\begin{minipage}{0.4\textwidth}
			\begin{flushright} \large
				Gonçalo Ribeiro, 73294
			\end{flushright}
		\end{minipage}\\[2.0cm]

		\large \textit{Docente: Prof. Horácio Neto}

		\vfill

		{\large \today}


	\end{center}

\end{titlepage}

\tableofcontents
\pagebreak

\section{Optimização do Algoritmo}
De maneira a minimizar o número de multiplicações utilizado no algoritmo, reorganizámos a expressão aritmeticamente.

\[
\left| A \right| = \left|
\begin{matrix}
a & b & c\\
d & e & f\\
g & h & i
\end{matrix}
\right| \Leftrightarrow\]

\[
\Leftrightarrow \left| A \right| = a e i - a f h - b d i + c d h + b f g - c e g \Leftrightarrow
\]

\[
\Leftrightarrow \left| A \right| = a \cdot ( e i - f h ) + b \cdot ( f g - d i ) + c \cdot ( d h - e g )
\]

Isto é, passámos de um algoritmo com 12 produtos para um algoritmo de 9 produtos.

\section{Escalonamento}

Na Figura~\ref{fig:fluxodados} pode ver-se o grafo de fluxo de dados correspondente ao algoritmo a implementar. Estão também identificados os caminhos críticos. Na Tabela~\ref{tab:listaprioridades} apresenta-se a lista de prioridades e na Tabela~\ref{tab:escalonamento2M1AS} o escalonamento ASAP (que acontece ser também ALAP) para 2 multiplicadores e 1 somador.

\begin{figure}[H]
\setlength{\unitlength}{0.75mm}
\centering
\begin{picture}(140,140)
% Operandos iniciais:
\put(15,130){\makebox(0,0){$e$}}
\put(25,130){\makebox(0,0){$i$}}
\put(35,130){\makebox(0,0){$f$}}
\put(45,130){\makebox(0,0){$h$}}
\put(55,130){\makebox(0,0){$f$}}
\put(65,130){\makebox(0,0){$g$}}
\put(75,130){\makebox(0,0){$d$}}
\put(85,130){\makebox(0,0){$i$}}
\put(95,130){\makebox(0,0){$d$}}
\put(105,130){\makebox(0,0){$h$}}
\put(115,130){\makebox(0,0){$e$}}
\put(125,130){\makebox(0,0){$g$}}
% Setas dos operandos para as operações:
\multiput(15,128)(20,0){6}{\vector(1,-2){2.4}}
\multiput(25,128)(20,0){6}{\vector(-1,-2){2.4}}
% Operações do primeiro nível:
\multiput(20,120)(20,0){6}{\circle{10}}
\multiput(20,120)(20,0){6}{\makebox(0,0){\texttt{x}}}
% Numeração das operações do primeiro nível:
\put(15,115){\makebox(0,0){\texttt{1}}}
\put(35,115){\makebox(0,0){\texttt{2}}}
\put(55,115){\makebox(0,0){\texttt{3}}}
\put(75,115){\makebox(0,0){\texttt{4}}}
\put(95,115){\makebox(0,0){\texttt{5}}}
\put(115,115){\makebox(0,0){\texttt{6}}}
% Setas para o segundo nível:
\multiput(20,115)(20,0){6}{\line(0,-1){15}}
\multiput(20,100)(40,0){3}{\vector(1,0){5}}
\multiput(40,100)(40,0){3}{\vector(-1,0){5}}
% Segundo nível de operações:
\multiput(30,100)(40,0){3}{\circle{10}}
\multiput(30,100)(40,0){3}{\makebox(0,0){\texttt{-}}}
% Numeração das operações do segundo nível:
\put(25,95){\makebox(0,0){\texttt{7}}}
\put(65,95){\makebox(0,0){\texttt{8}}}
\put(105,95){\makebox(0,0){\texttt{9}}}
% Setas para o terceiro nível:
\multiput(30,95)(40,0){3}{\vector(0,-1){10}}
\multiput(45,80)(40,0){3}{\vector(-1,0){10}}
% Terceiro nível de operações:
\multiput(30,80)(40,0){3}{\circle{10}}
\multiput(30,80)(40,0){3}{\makebox(0,0){\texttt{x}}}
% Operadores Constantes do Terceiro Nível:
\put(47,80){\makebox(0,0){$a$}}
\put(87,80){\makebox(0,0){$b$}}
\put(127,80){\makebox(0,0){$c$}}
% Numeração das operações do 3º nível:
\put(25,75){\makebox(0,0){\texttt{10}}}
\put(65,75){\makebox(0,0){\texttt{11}}}
\put(105,75){\makebox(0,0){\texttt{12}}}
% Setas para o 4º e 5º níveis
\multiput(30,75)(40,0){2}{\line(0,-1){15}}
\put(110,75){\line(0,-1){35}}
\put(30,60){\vector(1,0){15}}
\put(70,60){\vector(-1,0){15}}
\put(110,40){\vector(-1,0){35}}
% Quarto nível de operações
\put(50,60){\circle{10}}
\put(50,60){\makebox(0,0){\texttt{+}}}
% Numeração
\put(45,55){\makebox(0,0){\texttt{13}}}
% Seta para o 5º Nível
\put(50,55){\line(0,-1){15}}
\put(50,40){\vector(1,0){15}}
% Quinto nível:
\put(70,40){\circle{10}}
\put(70,40){\makebox(0,0){\texttt{+}}}
% Numeração
\put(65,35){\makebox(0,0){\texttt{14}}}
% Seta para o NOP:
\put(70,35){\vector(0,-1){10}}
% NOP:
\put(70,20){\circle{10}}
\put(70,20){\makebox(0,0){\texttt{NOP}}}
% Prioridades:
\put(75,45){\makebox(0,0){\textcolor{blue}{$1$}}}
\put(55,65){\makebox(0,0){\textcolor{green}{$2$}}}
\put(115,85){\makebox(0,0){\textcolor{green}{$2$}}}
\multiput(35,85)(40,0){2}{\makebox(0,0){\textcolor{yellow}{$3$}}}
\put(115,105){\makebox(0,0){\textcolor{yellow}{$3$}}}
\multiput(35,105)(40,0){2}{\makebox(0,0){\textcolor{orange}{$4$}}}
\multiput(105,125)(20,0){2}{\makebox(0,0){\textcolor{orange}{$4$}}}
\multiput(25,125)(20,0){4}{\makebox(0,0){\textcolor{red}{$5$}}}
\end{picture}
\caption{grafo de fluxo de dados e caminhos críticos}
\label{fig:fluxodados}
\end{figure}

\begin{table}
\centering
\begin{tabular}{|c|c||c l|}
\hline
Operação & Tipo & Prioridade & \\
\hline
\hline
\texttt{1} & \texttt{x} & \textcolor{red}{$5$} & \\
\hline
\texttt{2} & \texttt{x} & \textcolor{red}{$5$} & \\
\hline
\texttt{3} & \texttt{x} & \textcolor{red}{$5$} & \\
\hline
\texttt{4} & \texttt{x} & \textcolor{red}{$5$} & \\
\hline
\hline
\texttt{7} & \texttt{-} & \textcolor{orange}{$4$} & $ \leftarrow $ \texttt{1 , 2} \\
\hline
\texttt{8} & \texttt{-} & \textcolor{orange}{$4$} & $ \leftarrow $ \texttt{3 , 4} \\
\hline
\texttt{5} & \texttt{x} & \textcolor{orange}{$4$} & \\
\hline
\texttt{6} & \texttt{x} & \textcolor{orange}{$4$} & \\
\hline
\hline
\texttt{10} & \texttt{x} & \textcolor{yellow}{$3$} & $ \leftarrow $ \texttt{7} \\
\hline
\texttt{11} & \texttt{x} & \textcolor{yellow}{$3$} & $ \leftarrow $ \texttt{8} \\
\hline
\texttt{9} & \texttt{-} & \textcolor{yellow}{$3$} & $ \leftarrow $ \texttt{5 , 6} \\
\hline
\hline
\texttt{13} & \texttt{+} & \textcolor{green}{$2$} & $ \leftarrow $ \texttt{10 , 11} \\
\hline
\texttt{12} & \texttt{x} & \textcolor{green}{$2$} & $ \leftarrow $ \texttt{9} \\
\hline
\hline
\texttt{14} & \texttt{+} & \textcolor{blue}{$1$} & $ \leftarrow $ \texttt{12 , 13} \\
\hline
\end{tabular}
\caption{lista de prioridades correspondente à Figura~\ref{fig:fluxodados}}
\label{tab:listaprioridades}
\end{table}

\begin{table}[H]
\centering
\begin{tabular}{|c||c|c|c|}
\hline 
Ciclo & \multicolumn{2}{c|}{MUL} & ADD/SUB \\ 
\hline
\hline 
1 & \texttt{1} & \texttt{2} & - \\ 
\hline 
2 & \texttt{3} & \texttt{4} & \texttt{7} \\ 
\hline 
3 & \texttt{5} & \texttt{6} & \texttt{8} \\ 
\hline 
4 & \texttt{10} & \texttt{11} & \texttt{9} \\ 
\hline 
5 & \texttt{12} & - & \texttt{13} \\ 
\hline 
6 & - & - & \texttt{14} \\ 
\hline 
\end{tabular} 
\caption{escalonamento para 1 somador \\
e 2 multiplicadores}
\label{tab:escalonamento2M1AS}
\end{table}


\section{Partilha de Recursos}
\subsection{Partilha Óptima de Registos}
\label{subsec:partilha_optim_registos}

A Tabela~\ref{tab:binding_optim_reg} mostra a distribuição das operações $\mathrm{z_1}$ a $\mathrm{z_{13}}$ por registos tendo como prioridade reduzir o número de multiplexers à entrada dos registos. O número do ciclo indica no final de que ciclo é que o resultado tem que ser armazenado.

\begin{table}[H]
\centering
\begin{tabular}{|c||c|c|c|c|}
\hline 
Ciclo & $\mathrm{R_A}$ & $\mathrm{R_B}$ & $\mathrm{R_C}$ & $\mathrm{R_D}$ \\ 
\hline 
\hline
1 & M1 (z1) & M2 (z2) & - & - \\ 
\hline 
2 & M1 (z3) & M2 (z4) & A/S (z5) & - \\ 
\cline{1-3}\cline{5-5}
3 & M1 (z6) & M2 (z7) &  & A/S (z8) \\ 
\hline 
4 & M1 (z9) & M2 (z10) & A/S (z11) & - \\ 
\hline 
5 & - & M2 (z12) & A/S (z13) & - \\ 
\hline 
\end{tabular}
%\caption{distribuição das operações pelos operadores de forma a optimizar a partilha dos registos}
\caption{partilha óptima dos registos}
\label{tab:binding_optim_reg}
\end{table}

Uma vez atribuídos os resultados de cada operação aos vários registos faz-se a partilha dos operadores tentando minimizar o número de multiplexers à entrada dos mesmos (Tabela~\ref{tab:binding_operadores}). A partilha dos operadores está agora condicionada à partilha dos registos feita anteriormente.

\begin{table}[H]
\centering
\begin{tabular}{|c||cc|cc|cc|}
\hline 
Ciclo & \multicolumn{2}{c|}{M1} & \multicolumn{2}{c|}{M2} & \multicolumn{2}{c|}{A/S} \\ 
\hline 
\hline
1 & e & i & f & h & - & - \\ 
\hline 
2 & g & f & d & i & $\mathrm{R_A}$ & $\mathrm{R_B}$ \\ 
\hline 
3 & d & h & e & g & $\mathrm{R_A}$ & $\mathrm{R_B}$ \\ 
\hline 
4 & b & $\mathrm{R_D}$ & a & $\mathrm{R_C}$ & $\mathrm{R_A}$ & $\mathrm{R_B}$ \\ 
\hline 
5 & - & - & c & $\mathrm{R_C}$ & $\mathrm{R_A}$ & $\mathrm{R_B}$ \\ 
\hline 
6 & - & - & - & - & $\mathrm{R_C}$ & $\mathrm{R_B}$ \\ 
\hline 
\end{tabular}
%\caption{distribuição de operandos por operador e ciclo condicionada pela distribuição da Tabela~\ref{binding_optim_reg}}
\caption{partilha dos operadores condicionada à partilha da \mbox{Tabela~\ref{tab:binding_optim_reg}}}
\label{tab:binding_operadores}
\end{table}

Podemos observar que fazendo a partilha dos recursos desta forma necessitamos de:
\begin{itemize}
\item 1 multiplexer de 2 entradas;
\item 1 multiplexer de 5 entradas (1 de 2 entradas e 1 de 4);
\item 3 multiplexers de 4 entradas;
\end{itemize}
Isto é equivalente a 10 multiplexers de 2 entradas.

\subsection{Partilha Óptima de Operadores}

A Tabela~\ref{tab:binding_optim_oper} mostra a partilha de operadores escolhida tendo como prioridade reduzir o número de multiplexers à entrada dos operadores.

\begin{table}[H]
\centering
\begin{tabular}{|c||ccc|ccc|ccc|}
\hline 
Ciclo & \multicolumn{3}{c|}{M1} & \multicolumn{3}{c|}{M2} & \multicolumn{3}{c|}{A/S} \\ 
\hline 
\hline 
1 & e & i & (z1) & f & h & (z2) & - & - & - \\ 
\hline 
2 & d & i & (z4) & f & g & (z3) & $\mathrm{R_A}$ & $\mathrm{R_B}$ & (z5) \\
\hline 
3 & d & h & (z6) & e & g & (z7) & $\mathrm{R_A}$ & $\mathrm{R_B}$ & (z8) \\ 
\hline 
4 & b & $\mathrm{R_D}$ & (z9) & a & $\mathrm{R_C}$ & (z10) & $\mathrm{R_A}$ & $\mathrm{R_B}$ & (z11) \\ 
\hline 
5 & - & - & - & c & $\mathrm{R_C}$ & (z12) & $\mathrm{R_A}$ & $\mathrm{R_B}$ & (z13) \\ 
\hline 
6 & - & - & - & - & - & - & $\mathrm{R_A}$ & $\mathrm{R_B}$ & (R) \\ 
\hline 
\end{tabular} 
\caption{partilha óptima dos operadores}
\label{tab:binding_optim_oper}
\end{table}

Em paralelo à Tabela~\ref{tab:binding_optim_oper} construímos a Tabela~\ref{tab:binding_reg} em que se mostra em que registos se guardam os resultados das operações de cada ciclo, tentando minimizar o número de multiplexers à entrada dos registos.

\begin{table}[H]
\centering
\begin{tabular}{|c||c|c|c|c|}
\hline 
Ciclo & $\mathrm{R_A}$ & $\mathrm{R_B}$ & $\mathrm{R_C}$ & $\mathrm{R_D}$ \\ 
\hline 
\hline
1 & M1 (z1) & M2 (z2) & - & - \\ 
\hline 
2 & M2 (z3) & M1 (z4) & A/S (z5) & - \\ 
\cline{1-3}\cline{5-5}
3 & M1 (z6) & M2 (z7) &  & A/S (z8) \\ 
\hline 
4 & M2 (z10) & M1 (z9) & A/S (z11) & - \\ 
\hline 
5 & A/S (z13) & M2 (z12) & - & - \\ 
\hline 
\end{tabular}
\caption{partilha dos registos condicionada à Tabela~\ref{tab:binding_optim_oper}}
\label{tab:binding_reg}
\end{table}

Partilhando primeiro os operadores e restringindo a partilha dos registos a essa escolha são precisos:
\begin{itemize}
\item 1 multiplexer de 2 entradas;
\item 1 multiplexer de 4 entradas;
\item 4 multiplexers de 3 entradas (2 de 2 entradas).
\end{itemize}
Isto é equivalente a 11 multiplexers de 2 entradas. Esta alternativa é então pior do que a apresentada em \ref{subsec:partilha_optim_registos} (10 MUX de 2 entradas). Como tal vamos usar a segunda.

\section{Circuito}
\subsection{Datapath}

O circuito projectado tem por base a partilha de recursos descrita em \ref{subsec:partilha_optim_registos}. A datapath pode ser vista na Figura~\ref{fig:datapath}.

Como os elementos da matriz têm 16 bits e as multiplicações efectuadas são sempre entre elementos da matriz ou um elemento da matriz e um resultado intermédio de 32 bit os multiplicadores implementados têm uma entrada com 16 bits e outra com 32. O somador/subtractor é de 48 bits porque os seus operandos são resultados dos multiplicadores (48 bits). Por fim, todos os registos são de 48 bits de forma a acomodarem resultados com esse tamanho, excepto o registo $\mathrm{R_D}$ que tem apenas 32 bits porque o único valor que nele é armazenado corresponde à operação $\mathrm{z_8}$, que é uma subtracção de operandos com 32 bits.

\begin{figure}[]
\centering
•
\caption{•}
\label{fig:datapath}
\end{figure}

\subsection{Unidade de Controlo}

Os sinais de saída da unidade de controlo controlam os MUX à entrada dos operadores e dos registos e controlam os sinais de enable. Como sinais de entrada tem \texttt{data\_ready} --- que informa a unidade de controlo de quando tem à sua entrada os operandos adequados --- e \texttt{rst} --- o sinal de reset. A UC tem ainda um sinal de saída \texttt{out\_ready} que adicionámos para que seja possível assinalar o fim do cálculo de um determinante. A máquina de estados está patente na Figura~\ref{fig:statemachine}.

\begin{figure}
\centering
•
\caption{•}
\label{fig:statemachine}
\end{figure}

\subsection{Simulação}

\section{Leitura de Memória com Dois Portos}

\section{Pipelining}

\subsection{Adaptação da Arquitectura Prévia}

\subsection{Maximização do Throughput}
\begin{table}[h]
\centering
\begin{tabular}{|c||c|c|c|c|c|}
\hline
 & d & e & f & g & h  \\
\hline
\hline
e & $-$ & $\times$ & $\times$ & $\times$ & $\times$  \\
\hline
f & $-$ & $-$ & $\times$ & $\times$ & $\times$ \\
\hline
g & $-$ & \checkmark & \checkmark & $\times$ & $\times$  \\
\hline
h & \checkmark & $-$ & \checkmark & $-$ & $\times$ \\
\hline
i & \checkmark & \checkmark & $-$ & $-$ & $-$ \\
\hline
\end{tabular}
\caption{Produtos entre constantes a ler da memória}
\label{tab:produtos}
\end{table}

De maneira a reduzir ao máximo o número de ciclos que o algoritmo leva, é adequado agrupar os pares de leitura da memória de maneira a que o produto seja feito e armazenado ao mesmo tempo da leitura.

Uma operação é feita assim que os seus precedentes ficam disponíveis, quer via registos, quer via leitura directa da memória.

Uma vez que o número de operandos a ler é ímpar ($9$), e como o objectivo é maximizar o \emph{throughput}, haverá dois tipos distintos de execução.

\subsubsection{Execuções ímpares (1ª, 3ª, ...)}

\begin{enumerate}
\item %1
\begin{table}[H]
\centering
\begin{tabular}{l|c|c}
Operação & \multicolumn{2}{|c}{Operandos} \\
\hline
\texttt{READ} & $d$ & $h$ \\
\hline
\texttt{MUL} & $d$ & $h$ \\
\hline
\texttt{STORE} & $R_1$ & $d$ \\
\texttt{STORE} & $R_2$ & $h$ \\
\texttt{STORE} & $R_3$ & $dh$ \\
\end{tabular}
\caption{Operações do 1º ciclo de execução ímpar}
\label{tab:odd_1}
\end{table}

\item %2
\begin{table}[H]
\centering
\begin{tabular}{l|c|c}
Operação & \multicolumn{2}{|c}{Operandos} \\
\hline
\texttt{READ} & $i$ & $e$ \\
\hline
\texttt{MUL} & $i$ & $e$ \\
\texttt{MUL} & $i$ & $R_1$($d$) \\
\hline
\texttt{STORE} & $R_1$ & $di$ \\
\texttt{STORE} & $R_4$ & $e$ \\
\texttt{STORE} & $R_5$ & $ie$ \\
\end{tabular}
\caption{Operações do 2º ciclo de execução ímpar}
\label{tab:odd_2}
\end{table}

\item %3
\begin{table}[H]
\centering
\begin{tabular}{l|c|c}
Operação & \multicolumn{2}{|c}{Operandos} \\
\hline
\texttt{READ} & $f$ & $g$ \\
\hline
\texttt{MUL} & $f$ & $g$ \\
\texttt{MUL} & $f$ & $R_2$($h$) \\
\texttt{MUL} & $g$ & $R_4$($e$) \\
\hline
\texttt{STORE} & $R_6$ & $fg$ \\
\texttt{STORE} & $R_2$ & $fh$ \\
\texttt{STORE} & $R_4$ & $ge$ \\
\end{tabular}
\caption{Operações do 3º ciclo de execução ímpar}
\label{tab:odd_3}
\end{table}


\end{enumerate}

\subsubsection{Execuções pares (2ª, 4ª, ...)}

\end{document}
\documentclass[a4paper]{article}

\usepackage[utf8]{inputenc}
\usepackage{graphicx}

\begin{document}

\begin{titlepage}

	\begin{center}

		\includegraphics[width=6cm]{./title}\\[3cm]

		\textsc{\LARGE Projecto de Sistemas Digitais}\\[1.5cm]

		\textsc{\Large Laboratório 3}\\[1.5cm]


		{ \huge \bfseries Processador Dedicado para \\[3mm] Processamento de Imagens Binárias \\[3cm] }


		\noindent
		\begin{minipage}{0.4\textwidth}
			\begin{flushleft} \large
				Rafael Gonçalves, 73786
			\end{flushleft}
		\end{minipage}
		\begin{minipage}{0.4\textwidth}
			\begin{flushright} \large
				Gonçalo Ribeiro, 73294
			\end{flushright}
		\end{minipage}\\[2.0cm]

		\large \textit{Docente: Prof. Horácio Neto}

		\vfill

		{\large \today}


	\end{center}

\end{titlepage}

\tableofcontents
\pagebreak

\section{Arquitectura}
\subsection{Descrição}
\indent A arquitectura que decidimos implementar divide-se em duas secções distintas, como na maioria das arquitecturas dedicadas:
\begin{itemize}
\item Uma Unidade de Controlo, mais especificamente uma Máquina de Estados
\item Uma Datapath, cuja especificação corresponde à disponibilizada no enunciado
\end{itemize}
\indent De modo a melhor fazer a interface com os elementos da placa de desenvolvimento, as várias componentes do sistema foram projectados com vista a fazer a interface com os botões, interruptores e displays da Basys.\\
\indent Uma vez que da Datapath saem os valores dos registos, que posteriormente terão de ser seleccionados para ir para os displays de sete segmentos, é necessário um multiplexer no exterior da mesma, cujo sinal de selecção provém dos interruptores da placa de desenvolvimento.\\
\pagebreak
\subsection{Esquemas}
%Faltam aqui os esquemas, não sei se fazer esquemas no meu programa (dia) se incorporar os diagramas do RTL, mas acho que dia era melhor, menos técnico, mais abstracto
\pagebreak

\section{Datapath}
\subsection{Descrição}
\indent A decisão mais relevante na datapath foi a codificação de sinal utilizada nos registos e na Unidade Lógico-Aritmética.\\
\indent De modo a compreender intuitivamente os valores em uso, os displays devem mostrar os valores com uma codificação sinal-módulo. Assim, e uma vez que os valores dos displays originam nos registos da Datapath, decidimos que todo o armazenamento de valores deve ser feito em formato sinal-módulo.\\
\indent Esta decisão reflectiu-se não só nos formatos dos registos como também na ULA em si.\\
\subsection{Registos}
\indent Os registos foram especificados através das funcionalidades de abstracção da linguagem VHDL, isto é, definimos uma arquitectura de registo com número de bits arbitrário. O tamanho de cada instância é depois determinado no código de instanciação.\\
\subsection{Unidade Lógico-Aritmética}
\subsubsection{Somador}
\subsubsection{Multiplicador}
\subsubsection{Shift Right Aritmético}
\subsubsection{Função Lógica NOR}
\pagebreak
\subsection{Simulações}
\pagebreak


\section{Unidade de Controlo}
\subsection{Descrição}
\subsection{Máquina de Estados}
\subsection{Sinais de Controlo}
\pagebreak
\subsection{Simulações}
\pagebreak


\section{Simulações}

\end{document}
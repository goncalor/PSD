\documentclass[a4paper]{article}

\usepackage[utf8]{inputenc}
\usepackage{graphicx}

\begin{document}

\begin{titlepage}

	\begin{center}

		\includegraphics[width=6cm]{./title}\\[3cm]

		\textsc{\LARGE Projecto de Sistemas Digitais}\\[1.5cm]

		\textsc{\Large Laboratório 3}\\[1.5cm]


		{ \huge \bfseries Processador Dedicado para \\[3mm] Processamento de Imagens Binárias \\[3cm] }


		\noindent
		\begin{minipage}{0.4\textwidth}
			\begin{flushleft} \large
				Rafael Gonçalves, 73786
			\end{flushleft}
		\end{minipage}
		\begin{minipage}{0.4\textwidth}
			\begin{flushright} \large
				Gonçalo Ribeiro, 73294
			\end{flushright}
		\end{minipage}\\[2.0cm]

		\large \textit{Docente: Prof. Horácio Neto}

		\vfill

		{\large \today}


	\end{center}

\end{titlepage}

\tableofcontents
\pagebreak

\section{Arquitectura}
\subsection{Descrição}
A arquitectura que decidimos implementar divide-se em duas secções distintas, como na maioria das arquitecturas dedicadas:
\begin{itemize}
\item Uma Unidade de Controlo, mais especificamente uma Máquina de Estados
\item Uma Datapath, cuja especificação corresponde à disponibilizada no enunciado
\end{itemize}

De modo a melhor fazer a interface com os elementos da placa de desenvolvimento, as várias componentes do sistema foram projectados com vista a fazer a interface com os botões, interruptores e displays da Basys.

Uma vez que da Datapath saem os valores dos registos, que posteriormente terão de ser seleccionados para ir para os displays de sete segmentos, é necessário um multiplexer no exterior da mesma, cujo sinal de selecção provém dos interruptores da placa de desenvolvimento.

\pagebreak
\subsection{Esquemas}
%Faltam aqui os esquemas, não sei se fazer esquemas no meu programa (dia) se incorporar os diagramas do RTL, mas acho que dia era melhor, menos técnico, mais abstracto
\pagebreak

\section{Datapath}
\subsection{Descrição}
A decisão mais relevante na datapath foi a codificação de sinal utilizada nos registos e na Unidade Lógico-Aritmética.

De modo a compreender intuitivamente os valores em uso, os displays devem mostrar os valores com uma codificação sinal-módulo. Assim, e uma vez que os valores dos displays originam nos registos da Datapath, decidimos que todo o armazenamento de valores deve ser feito em formato sinal-módulo.


\subsection{Registos}
Os registos foram especificados através das funcionalidades de abstracção da linguagem VHDL, isto é, definimos uma arquitectura de registo com número de bits arbitrário. O tamanho de cada instância é depois determinado no código de instanciação.

Neste caso, foram instanciados dois registos, um de sete bits e outro de 13 bits, consoante a arquitectura especificada no enunciado.


\subsection{Unidade Lógico-Aritmética}
Uma vez que os dois operadores das diversas unidades funcionais que compõem a ULA estão codificados no formato sinal-módulo, é necessário fazer a conversão de sinal-módulo para complemento para dois.

O Somador (que efectua as operações soma e diferença) e o Multiplicador foram incluídos da biblioteca \emph{signed}, pelo que os seus operandos devem ser convertidos em complemento para dois à entrada, e o resultado deve ser convertido para sinal-módulo à saída.


\subsubsection{Somador}
O Somador efectua uma soma ou diferença, consoante os sinais de controlo, convencional, em complemento para dois.

\subsubsection{Multiplicador}
O multiplicador efectua um produto convencional em complemento para dois. Uma vez que o produto ocupa mais bits que aqueles disponíveis no registo que o deve armazenar, são desprezados vários dos bits do resultado.

\subsubsection{Shift Right Aritmético}
A função shift afecta apenas os bits de módulo do valor armazenado no registo 2. O bit de sinal permanece inalterado.

\subsubsection{Função Lógica NOR}
A função lógica NOR aplica-se aos bits de sinal dos dois operadores e aos 6 bits menos significativos dos dois operadores. Os restantes bits do segundo operador (de dimensão 13) são negados (isto é, NOR 0).

\pagebreak
\subsection{Simulações}
\pagebreak


\section{Unidade de Controlo}
\subsection{Descrição}
\subsection{Máquina de Estados}
\subsection{Sinais de Controlo}
\pagebreak
\subsection{Simulações}
\pagebreak


\section{Simulações}

\end{document}
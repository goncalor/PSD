\documentclass[a4paper]{article}

\usepackage[portuguese]{babel}
\usepackage{comment}
\usepackage[T1]{fontenc}
\usepackage[utf8]{inputenc}
\usepackage{hyperref}
\usepackage{graphicx}
\usepackage{float}
\usepackage{multirow}
\usepackage[hypcap]{caption} % makes \ref point to top of figures and tables
\usepackage{amsmath}
\usepackage{multicol} %for page layout on section Maximização do Throughput
%\usepackage[usenames,dvipsnames,svgnames,table]{xcolor}
\usepackage{pdflscape}	% landscape pages
\usepackage{subcaption}

\title{Projecto 3 --- Adenda}
\author{Gonçalo Ribeiro, 73294\hspace{8mm}Rafael Gonçalves, 73786}

\begin{document}
\maketitle
%\begin{titlepage}

	\begin{center}

		\includegraphics[width=6cm]{./title}\\[3cm]

		\textsc{\LARGE Projecto de Sistemas Digitais}\\[1.5cm]

		\textsc{\Large Laboratório 3}\\[1.5cm]


		{ \huge \bfseries Processador Dedicado para \\[3mm] Processamento de Imagens Binárias \\[3cm] }


		\noindent
		\begin{minipage}{0.4\textwidth}
			\begin{flushleft} \large
				Rafael Gonçalves, 73786
			\end{flushleft}
		\end{minipage}
		\begin{minipage}{0.4\textwidth}
			\begin{flushright} \large
				Gonçalo Ribeiro, 73294
			\end{flushright}
		\end{minipage}\\[2.0cm]

		\large \textit{Docente: Prof. Horácio Neto}

		\vfill

		{\large \today}


	\end{center}

\end{titlepage}

%\tableofcontents

\section{Resumo}
O presente documento é uma adenda ao relatório entregue no dia 21 de Dezembro de 2014.

À data de entrega do relatório a unidade HALF estava concluída e do SAVE faltava afinar a máquina de estados. Faltava também ligar as BRAMs, e fazer a máquina de estados que pede input ao utilizador.

Os pontos que se seguem resumem as alterações que foram feitas em relação à data de entrega do relatório.

\begin{itemize}
	\item redesenhou-se a unidade de armazenamento;
	\item adicionou-se uma segunda FSM ao módulo SAVE, que permite tratar correctamente as operações compostas;
	\item desenhou-se a FSM que permite ao utilizador introduzir os parâmetros largura, altura e tipo de operação;
	\item fizeram-se as ligações às BRAMs.
\end{itemize}

\section{Detalhes das Alterações}

\subsection*{Unidade de Armazenamento}

\subsection*{FSM do SAVE}

\subsection*{FSM de Input}

\subsection*{Ligação às BRAMs}

\end{document}

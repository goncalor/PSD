\documentclass[a4paper]{article}

\usepackage[portuguese]{babel}
\usepackage{comment}
\usepackage[T1]{fontenc}
\usepackage[utf8]{inputenc}
\usepackage{hyperref}
\usepackage{graphicx}
\usepackage{float}
\usepackage{multirow}
\usepackage[hypcap]{caption} % makes \ref point to top of figures and tables
\usepackage{amsmath}
\usepackage{multicol} %for page layout on section Maximização do Throughput
%\usepackage[usenames,dvipsnames,svgnames,table]{xcolor}
\usepackage{pdflscape}	% landscape pages

\begin{document}

\begin{titlepage}

	\begin{center}

		\includegraphics[width=6cm]{./title}\\[3cm]

		\textsc{\LARGE Projecto de Sistemas Digitais}\\[1.5cm]

		\textsc{\Large Laboratório 3}\\[1.5cm]


		{ \huge \bfseries Processador Dedicado para \\[3mm] Processamento de Imagens Binárias \\[3cm] }


		\noindent
		\begin{minipage}{0.4\textwidth}
			\begin{flushleft} \large
				Rafael Gonçalves, 73786
			\end{flushleft}
		\end{minipage}
		\begin{minipage}{0.4\textwidth}
			\begin{flushright} \large
				Gonçalo Ribeiro, 73294
			\end{flushright}
		\end{minipage}\\[2.0cm]

		\large \textit{Docente: Prof. Horácio Neto}

		\vfill

		{\large \today}


	\end{center}

\end{titlepage}

\tableofcontents
\pagebreak

\section{Introdução}

\section{Formato das Imagens}

\section{Arquitectura}
% intro à arquitectura

\subsection{HALF}
% intro ao HALF

\subsubsection{Datapath}

\subsubsection{FSM}

\subsection{SAVE}
% intro ao SAVE

\subsubsection{Datapath}

\subsubsection{FSM}

\subsection{\textit{Padder}}
% intro ao padder

\subsubsection{Datapath}

\subsubsection{FSM}

\subsection{Interligação dos Componentes}	% titulo melhor ?

\section{\textit{Performance}}	% titulo melhor ?

\section{Conclusões e Comentários}

\end{document}
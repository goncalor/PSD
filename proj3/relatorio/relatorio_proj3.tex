\documentclass[a4paper]{article}

\usepackage[portuguese]{babel}
\usepackage{comment}
\usepackage[T1]{fontenc}
\usepackage[utf8]{inputenc}
\usepackage{hyperref}
\usepackage{graphicx}
\usepackage{float}
\usepackage{multirow}
\usepackage[hypcap]{caption} % makes \ref point to top of figures and tables
\usepackage{amsmath}
\usepackage{multicol} %for page layout on section Maximização do Throughput
%\usepackage[usenames,dvipsnames,svgnames,table]{xcolor}
\usepackage{pdflscape}	% landscape pages

\begin{document}

\begin{titlepage}

	\begin{center}

		\includegraphics[width=6cm]{./title}\\[3cm]

		\textsc{\LARGE Projecto de Sistemas Digitais}\\[1.5cm]

		\textsc{\Large Laboratório 3}\\[1.5cm]


		{ \huge \bfseries Processador Dedicado para \\[3mm] Processamento de Imagens Binárias \\[3cm] }


		\noindent
		\begin{minipage}{0.4\textwidth}
			\begin{flushleft} \large
				Rafael Gonçalves, 73786
			\end{flushleft}
		\end{minipage}
		\begin{minipage}{0.4\textwidth}
			\begin{flushright} \large
				Gonçalo Ribeiro, 73294
			\end{flushright}
		\end{minipage}\\[2.0cm]

		\large \textit{Docente: Prof. Horácio Neto}

		\vfill

		{\large \today}


	\end{center}

\end{titlepage}

\tableofcontents
\pagebreak

\section{Introdução}
O presente trabalho tem por objectivo o desenho e implementação de um processador dedicado para processamento morfológico de imagens binárias i.e.\ imagens cujos pixeis podem ter apenas dois valores distintos.

As operações a realizar sobre as imagens são cinco: erosão, dilatação, fecho (dilatação seguida de erosão), abertura (erosão seguida de dilatação) e extração de contornos (diferença entre a imagem original e a sua erosão).

O elemento estruturante considerado é um quadrado de $3\times3$ pixeis---por exemplo na operação de erosão analisa-se uma área $3\times3$ px centrada nesse pixel para decidir se o mesmo deve ser erodido.

As imagens consideradas têm no máximo $128\times128$ px. Uma imagem é transferida de um computador para a uma BRAM da FPGA via USB e o processador dedicado deve ler dessa BRAM de entrada e escrever a imagem transformada numa BRAM de saída. A imagem transformada é depois transferida de volta PC, onde pode ser visualizada.

O desempenho do circuito é avaliado pelo tempo de processamento dos dados entre as duas memórias.

\section{Formato das Imagens}
O formato das imagens não é definido no enunciado do projecto. É esperado que os alunos desenvolvam um formato de imagem que julguem mais adequado.

As imagens têm no máximo $128\times128\times1$ bits (2 kB), que é precisamente a dimensão de cada uma das 4 BRAMs existentes na FPGA XC3S-100E-4CP132. Como tal no caso das imagens maiores a BRAM de entrada é completamente preenchida, o que significa que não sobram bits para representar por exemplo as dimensões da imagem. Assim, essa informação não pode fazer parte do formato da imagem.

Decidiu-se então que as dimensões da imagem (largura e altura) são passadas ao processador dedicado fazendo uso dos 8 interruptores existentes na placa de desenvolvimento. Introduz-se primeiro a largura e pressiona-se um botão de pressão para armazenar essa dimensão. De seguida faz-se o mesmo para a altura.

A imagem é representada em memória sempre como 4 palavras de 32 bits por cada linha da imagem. Se a imagem tiver menos que 128 bits de largura é introduzido \textit{padding} que perfaça 128 bits. O \textit{padding} utilizado consiste na extensão do último bit válido de cada linha da imagem. Estas características foram escolhida de forma a facilitar o processamento da imagem.

\section{Arquitectura}
% intro à arquitectura

\subsection{HALF}
% intro ao HALF

\subsubsection{Datapath}

\subsubsection{FSM}

\subsection{SAVE}
% intro ao SAVE

\subsubsection{Datapath}

\subsubsection{FSM}

\subsection{\textit{Padder}}
% intro ao padder

\subsubsection{Datapath}

\subsubsection{FSM}

\subsection{Interligação dos Componentes}	% titulo melhor ?

\section{\textit{Performance}}	% titulo melhor ?

\section{Conclusões e Comentários}

\end{document}